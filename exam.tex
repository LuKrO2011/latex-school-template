\documentclass[points=left, 
solution,
textsize=12pt,
mathsize=14pt,
boxoffset=0.05cm
]{exam}

\date{19.03.2025}
\title{2. Schulaufgabe aus der Mathematik}
\class{Klasse 8b}
\notes{
    \begin{itemize}
        \item Zeit: 45 min
        \item Erlaubtes Hilfsmittel: Taschenrechner
        \item Kürze vollständig!
        \item Achte auf eine saubere und nachvollziehbare Darstellung der Rechenschritte!
    \end{itemize}
}

\usepackage{lipsum}

\date{01.09.2024}
\title{Lösen von Exponentialgleichungen}
\class{5A}
\notes{
    \begin{itemize}
        \item Kürze vollständig
    \end{itemize}
}

\begin{document}

\task{9}

\lipsum[1]

% Use dfrac for fractions to ensure proper spacing
\begin{enumerate}[label=\alph*), left=0pt]
    \item Löse die Gleichung $\dfrac{4}{ x} - \dfrac{2}{x+2}$
\end{enumerate}

\squares{5}

\begin{enumerate}[label=\alph*), resume, left=0pt]
    \item Löse die Gleichung \(x^2 - 9 = 0\).
\end{enumerate}

\squares{8}

\begin{enumerate}[label=\alph*), resume, left=0pt]
    \item Löse die Gleichung \(x^2 - 16 = 0\).
\end{enumerate}

\squares{5}

\task{5}

Löse die Gleichung \(2x + 3x - 5\).
\squares{10}

\task{7}

Löse die Gleichung:
% Centered equation
\[
    2x + 3x - 5 = 0
\]
\thinkbubble[Test]
\squares{15}

\solution{
    Hallo
}

\totalpoints

\end{document}
