\documentclass[points=left, 
solution,
textsize=14pt,
mathsize=16pt,
boxoffset=0.67cm
]{exam}

\usepackage{lipsum}

\date{01.09.2024}
\title{Lösen von Exponentialgleichungen}
\class{5A}
\notes{
    \begin{itemize}
        \item Kürze vollständig
    \end{itemize}
}

\begin{document}

\task{10}

\lipsum[1]

\begin{enumerate}[label=\alph*)]
    \item Löse die Gleichung \(x^2 - 4 = 0\). \\
\end{enumerate}

\squares{5}

\begin{enumerate}[label=\alph*), resume]
    \item Löse die Gleichung \(x^2 - 9 = 0\).
\end{enumerate}

\squares{5}

\begin{enumerate}[label=\alph*), resume]
    \item Löse die Gleichung \(x^2 - 16 = 0\).
\end{enumerate}

\squares{5}

\task{5}

\lipsum[1]

Löse die Gleichung \(2x + 3x - 5\).
\squares{14}

\task{10}
Löse die Gleichung:
% Centered equation
\[
    2x + 3x - 5 = 0
\]
\thinkbubble[Test]
\squares{15}

\solution{
    Hallo
}

\totalpoints

\end{document}
